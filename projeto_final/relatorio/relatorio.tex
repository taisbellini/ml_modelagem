% abtex2-modelo-artigo.tex, v-1.9.2 laurocesar
% Copyright 2012-2014 by abnTeX2 group at http://abntex2.googlecode.com/ 
%
% abnTeX2: Modelo de Artigo Acadêmico em conformidade com
% ABNT NBR 6022:2003: Informação e documentação - Artigo em publicação 
% periódica científica impressa - Apresentação
%-------------------------------------------------------------------------------------------

\documentclass[
	% -- opções da classe memoir --
	article,			% indica que é um artigo acadêmico
	11pt,				% tamanho da fonte
	oneside,			% para impressão apenas no verso. Oposto a twoside
	a4paper,			% tamanho do papel. 
	% -- opções da classe abntex2 --
	%chapter=TITLE,		% títulos de capítulos convertidos em letras maiúsculas
	%section=TITLE,		% títulos de seções convertidos em letras maiúsculas
	%subsection=TITLE,	% títulos de subseções convertidos em letras maiúsculas
	%subsubsection=TITLE % títulos de subsubseções convertidos em letras maiúsculas
	% -- opções do pacote babel --
	english,			% idioma adicional para hifenização
	brazil,				% o último idioma é o principal do documento
	sumario=tradicional
	]{abntex2}

\usepackage{lmodern}			% Usa a fonte Latin Modern
\usepackage[T1]{fontenc}		% Selecao de codigos de fonte.
\usepackage[utf8]{inputenc}		% Codificacao do documento (conversão automática dos acentos)
\usepackage{indentfirst}		% Indenta o primeiro parágrafo de cada seção.
\usepackage{nomencl} 			% Lista de simbolos
\usepackage{color}				% Controle das cores
\usepackage{graphicx}			% Inclusão de gráficos
\usepackage{microtype} 			% para melhorias de justificação
\usepackage{lipsum}				% para geração de dummy text
\usepackage[brazilian,hyperpageref]{backref}	 % Paginas com as citações na bibl
\usepackage[num]{abntex2cite}	% Citações padrão ABNT \alf
\usepackage{authblk}
\usepackage{amsmath}
\usepackage{amsfonts}

\renewcommand{\backrefpagesname}{Citado na(s) página(s):~}
% Texto padrão antes do número das páginas
\renewcommand{\backref}{}
% Define os textos da citação
\renewcommand*{\backrefalt}[4]{
	\ifcase #1 %
		Nenhuma citação no texto.%
	\or
		Citado na página #2.%
	\else
		Citado #1 vezes nas páginas #2.%
	\fi}%

% Informações de dados

\titulo{Título do Trabalho \\ Subtítulo do Trabalho}

\author[1]{Albertine Weber}
\author[2]{Brenda Rolt}
\author[1]{Eduardo Brock}
\author[2]{Mayara Soares}
\author[2]{Tais Bellini}

\affil[1]{Instituto de Física - Universidade Federal do Rio Grande do Sul}
\affil[2]{Instituto de Matemática e Estatística - Universidade Federal do Rio Grande do Sul}

\local{Brasil}

\data{\today}

% Configurações de aparência do PDF final

\definecolor{blue}{RGB}{41,5,195}
\makeatletter
\hypersetup{
     	%pagebackref=true,
		pdftitle={\@title}, 
		pdfauthor={\@author},
    	pdfsubject={Modelo de artigo científico com abnTeX2},
	    pdfcreator={LaTeX with abnTeX2},
		pdfkeywords={abnt}{latex}{abntex}{abntex2}{atigo científico}, 
		colorlinks=true,       		% false: boxed links; true: colored links
    	linkcolor=blue,          	% color of internal links
    	citecolor=blue,        		% color of links to bibliography
    	filecolor=magenta,      	% color of file links
		urlcolor=blue,
		bookmarksdepth=4
}

\makeatother
\setlrmarginsandblock{3cm}{3cm}{*} % Altera as margens padrões
\setulmarginsandblock{3cm}{3cm}{*} % Altera as margens padrões
\checkandfixthelayout
\makeindex % compila o indice
\setlength{\parindent}{1.3cm} % O tamanho do parágrafo
\setlength{\parskip}{0.2cm}  % Controle do espaçamento entre um parágrafo e outro
\SingleSpacing % Espaçamento simples
%-------------------------------------------------------------------------------------------

\begin{document}
\selectlanguage{brazil}
% ----------------------------------------------------------
% Capa
% ----------------------------------------------------------

\begin{titlepage}
    \begin{center}
        \Large
        UNIVERSIDADE FEDERAL DO RIO GRANDE DO SUL\\
        INSTITUTO DE MATEMÁTICA E ESTATÍSTICA\\
        
        \vspace{1.5cm}
 
        Albertine Weber\\
        Brenda Rolt\\
        Eduardo Brock\\
        Mayara Soares\\
        Tais Bellini
        
        \vspace{5.0cm}
        
        \Huge\textbf{Título do Trabalho}
        
        \vspace{0.5cm}
        
        \LARGE Subtítulo do Trabalho
        
        \vfill
        
        Projeto final do curso de Machine Learning e Modelagem Estatística ofertado durante o Curso de Verão 2020
 
        \vspace{0.8cm}
 
        \Large
        Porto Alegre\\
        \today
    \end{center}
\end{titlepage}

\pretextual
\frenchspacing % Retira espaço extra obsoleto entre as frases.
\maketitle

\begin{resumoumacoluna}
\lipsum[1]
 
 \vspace{\onelineskip}
 
 \noindent
 \textbf{Palavras-chaves}: PALAVRA-CHAVE, PALAVRA-CHAVE E PALAVRA-CHAVE
\end{resumoumacoluna}

\textual
% ----------------------------------------------------------
% Introdução
% ----------------------------------------------------------
\section*{Introdução}
\addcontentsline{toc}{section}{Introdução}


Paper \cite{fehrman2015}

% ----------------------------------------------------------
% Cap. 1
% ----------------------------------------------------------
\section{Introdução}

Introducao - Descricao superficial do problema:

- Classificar em usuario e nao usuario;

Não usuário: Never Used, Over a Decade ago, Last Decade

Usuário: Last Year, Last Month, Last Week, Last Day

- Se basear nas variaveis...

- Analise sera feita para 3 drogas e um grupo:

Drogas: Alcool, Ecstasy, Cannabis

Grupo de Estimulantes: Cocaína, Crack, Anfetamina

- Quais drogas escolhemos analisar: Alcool, ecstasy, Cannabis, grupo. Justificar

- Indicar que avaliaremos acuracia e sensibilidade, pois queremos acertar os usuarios.

- Faremos a analise para cada droga/grupo pois como as disrbuicoes e comportamentos diferem pode ser que mude.

- Métricas para avaliar o melhor: acurácia e sensibilidae (explicar).

% ----------------------------------------------------------
% Cap. 2
% ----------------------------------------------------------
\section{Descrição dos dados}

- Resumo geral: numero de linhas, mostrar todas as variaveis disponiveis

- Indicar que pode ser viesado pela forma de coleta de dados

\subsection{Resumo dos dados}

- Algumas estatisticas

\subsection{Variáveis excluídas a priori}

- Country, Ethnicity: mostrar distribuição, que tem muito UK, muito white e que pode interferir na analise.

- Referenciar estudo

- Impulsive e SS: Nao foram encontradas referencias confiaveis sobre como classificar os niveis

\subsection{Variáveis que serão utilizadas}

- Scores: explicar o que cada um significa

- Explicar a classificacao em very low, low, average, high, very high

- Age, gender, education ... mostrar categorias, distribuicoes e etc. separado por droga/grupo

- Lista de drogas disponiveis e os seus níveis

\subsection{Distribuicoes das drogas escolhidas e variaveis associadas}

% ----------------------------------------------------------
% Cap. 3
% ----------------------------------------------------------

\section{Técnica empregada}

O objetivo desde estudo é criar um classificador que, baesado das nas variáveis descritas no capítulo 2, indica se aquela pessoa tem mais chance de ser usuária de uma determinada droga ou grupo de drogas ou não. 

Para realizar a classificação, consideramos usuários os indivíduos que tinham como resposta Never Used, Used a Decade Ago e Used in Last Decade. Os demais foram considerados usuários. Esta escolha foi feita a partir de um estudo prévio (ref).

Foram availiados X métodos: Regressao Logística, arvore de decisao, ... 

Em cada método, para cada droga ou grupo de drogas, foi avaliada a possibilidade de reduzir o número de variáveis independentes de forma que mantivesse ou melhorasse o resultado. Compararemos os métodos com o mesmo subset de variáveis independentes para definir qual será melhor de acordo com as métricas acurácia e sensibilidade. Apresentaremos os resultados finais do método escolhido. 

\subsection{Regressão Logística}

Para a estimação do modelo de regressão logística foi utilizado o pacote \textbf{glm} do R com o parâmetro \emph{family} como \emp{binomial}.
Para cada droga ou grupo de droga, inicialmente, geramos um modelo utilizando todas as variáveis e obtivemos \textbf{acurácia} e \textbf{sensibilidade}. Depois, foi aplicada a função \textbf{StepAIC} do pacote \textbf{MASS} para determinar se um modelo com menos variáveis seria mais ou igualmente eficiente de acordo com as métricas escolhidas.



\section{Resultados}

\subsection{Cannabis}

retomar caracteristicas da amostra
 0   1 
886 999 

\subsubsection{Reg logistica}

No caso da regressão logística, utilizando todas as variáveis e o ponto de corte em 0.5, estimamos um modelo com 78.7\% de acurácia, e 84.06\% de sensibilidade. 
Utilizando a função \textbf{stepAIC} com o parâmetro \emph{direction} definido como \emph{both}, temos um modelo que considera apenas as variáveis independente \textbf{Age, Gender, Education, NScore, OScore, CScore} sem alterar o resultado das métricas. Assim, observamos que EScore(botar o nome certo) e AScore (botar o nome certo) podem ter pouca influência no uso de maconha.
Alterando o ponto de corte para 0.8, obtivemos uma acurácia de 79.43\% e sensibilidade de 89.86\%. (colocar possivel explicacao)


\subsubsection{Arvore de decisao}

... 

\subsubsection{comparacao e escolha do modelo e metodo}

\subsection{Ecstasy}

retomar caracteristicas da amostra

\subsubsection{Reg logistica}

- resumido, nao apresentar tudo, apenas o que for relevante.

- comparar full com o reduzido com AIC

- importancia das variaveis

- confusion matriz e resultados (acc, sensibilidade)

\subsubsection{Arvore de decisao}

... 

\subsubsection{comparacao e escolha do modelo e metodo}

- comparar

- escolher o melhor e executar resultados finais (acuracia, sensibilidade, ...)

\subsection{Alcool}

Foi observado que a amostra utilizada para avaliar se um indivíduo é ou não usuário de Álcool não tem boas propriedades e os modelos estimados não possuem boas propriedades. Apenas 7\% dos indivíduos da amostra não são usuários, o que ...

- graficos com variaveis independentes e classificacao usuario e nao usuario 

\subsubsection{Reg logistica}

No caso da regressão logística, utilizando todas as variáveis e o ponto de corte em 0.5, estimamos um modelo com 94.3\% de acurácia, mas 0\% de sensibilidade. Alterando o ponto de corte para 0.3 ou 0.8 não apresentou diferença no resultado. Isto porque como a grande maioria das pessoas da amostra são usuários, o algoritmo classifica sempre como usuário e tem boas chances de acertar. 
Utilizando a função \textbf{stepAIC} com o parâmetro \emph{direction} definido como \emph{both}, temos um modelo que considera apenas a variável independente \textbf{Age}.
Talvez fazer alguma coisa para eventos raros (se der tempo, se nao so comentar que seria uma solucao).


- comparar full com o reduzido com AIC

- importancia das variaveis

- confusion matriz e resultados (acc, sensibilidade)
\subsubsection{Arvore de decisao}
... 

\subsubsection{comparacao e escolha do modelo e metodo}

- comparar

- escolher o melhor e executar resultados finais (acuracia, sensibilidade, ...)

\subsection{Grupo}

Retomar caracteristica da amostra

\subsubsection{Reg logistica}

- resumido, nao apresentar tudo, apenas o que for relevante.

- comparar full com o reduzido com AIC

- importancia das variaveis

- confusion matriz e resultados (acc, sensibilidade)

\subsubsection{Arvore de decisao}
... 

\subsubsection{comparacao e escolha do modelo e metodo}

- comparar

- escolher o melhor e executar resultados finais (acuracia, sensibilidade, ...)

% ---
% Finaliza a parte no bookmark do PDF, para que se inicie o bookmark na raiz
% ---
\bookmarksetup{startatroot}% 

% ----------------------------------------------------------
% Conclusão
% ----------------------------------------------------------
\section*{Considerações finais}

\addcontentsline{toc}{section}{Considerações finais}

- apos avaliar os metodos X, Y, Z, observamos que, para estes dados, o mais adequado foi... Que para a droga tal e tal o modelo tal ficou melhor, ...

- retomar problemas do banco de dados

- indicar outras coisas que poderiam ser feitas: tratar nao usuario de alcool como evento raro, ...

% ----------------------------------------------------------
% Referências bibliográficas
% ----------------------------------------------------------
\bibliography{referencias}

% ----------------------------------------------------------
% Glossário
% ----------------------------------------------------------
%
% Há diversas soluções prontas para glossário em LaTeX. 
% Consulte o manual do abnTeX2 para obter sugestões.
%
%\glossary

% ----------------------------------------------------------
% Apêndices
% ----------------------------------------------------------
%
%\begin{apendicesenv}
%
% ----------------------------------------------------------
%\chapter{Apêndice}
% ----------------------------------------------------------
%\lipsum[55-57]
%
%\end{apendicesenv}

% ----------------------------------------------------------
% Anexos
% ----------------------------------------------------------
%\cftinserthook{toc}{AAA}
%\anexos
%\begin{anexosenv}
%
%\chapter{Anexo}
%
%\lipsum[31]
%\end{anexosenv}

\end{document}